\selectlanguage{french}

\def\Title{Le contrat social Debian}

\def\Version{
Version~1.1 ratifiée le 26~avril~2004. Remplace la version~1.0 ratifiée le
5~juillet~1997.
}

\def\Intro{
Le projet Debian, concepteur du système Debian, a créé le \textbf{contrat
social de Debian}. La partie consacrée aux principes du logiciel libre selon
Debian (\textbf{Debian Free Software Guidelines} ou DFSG), initialement conçue
comme un ensemble de principes auxquels nous tenons fermement, a été adoptée
par la communauté du logiciel libre comme base pour la définition de
l'informatique libre (\textbf{Open Source Definition}).
}

% ****** Social Contract ******

\def\SCT{\emph{«~Contrat social~»} avec la communauté des logiciels libres}

\def\SCa{Debian demeurera totalement libre.}

\def\SCat{
Le document intitulé \emph{«~principes du logiciel libre selon Debian~»}
énonce les critères grâce auxquels le projet Debian détermine si un travail
est \emph{«~libre~»}. Nous promettons que le système Debian et tous ses
composants seront libres selon ces principes et nous aiderons les personnes qui
créent et utilisent à la fois des travaux libres et non libres sur Debian.
Nous ne rendrons pas le système dépendant d'un composant non libre.
}

\def\SCb{Nous donnerons nos travaux à la communauté des logiciels libres.}

\def\SCbt{
Lorsque nous réaliserons de nouveaux composants du système Debian, nous les
publierons sous une licence compatible avec les principes du logiciel libre
selon Debian. Nous ferons le meilleur système possible, afin que les travaux
libres soient largement distribués et utilisés. Nous signalerons les
corrections de bogues, les améliorations, les requêtes des utilisateurs, etc.
aux auteurs \emph{«~amont~»} des travaux originels inclus dans notre système.
}

\def\SCc{Nous ne dissimulerons pas les problèmes.}

\def\SCct{
Nous conserverons l'intégralité de notre base de données de rapports de bogue
accessible au public en tout temps. Les rapports que les utilisateurs
remplissent en ligne seront rapidement visibles par les autres.
}

\def\SCd{Nos priorités sont nos utilisateurs et les logiciels libres.}

\def\SCdt{
Les besoins de nos utilisateurs et de la communauté des logiciels libres nous
guideront. Nous placerons leurs intérêts en tête de nos priorités. Nous
répondrons aux besoins de nos utilisateurs dans de nombreux types
d'environnements informatiques différents. Nous ne nous opposerons pas aux
travaux non libres prévus pour fonctionner sur les systèmes Debian. Nous
permettrons, sans réclamer rétribution, que d'autres créent des distributions
contenant conjointement des logiciels Debian et d'autres travaux. Pour servir
ces objectifs, nous fournirons un système intégrant des composants de grande
qualité sans restrictions légales incompatibles avec ces modes d'utilisation.
}

\def\SCe{Travaux non conformes à nos standards sur les logiciels libres.}

\def\SCet{
Nous reconnaissons que certains de nos utilisateurs demandent à pouvoir
utiliser des travaux qui ne sont pas conformes aux principes du logiciel libre
selon Debian. Les paquets correspondant prennent place dans des sections
nommées \emph{«~contrib~»} («~contributions~») et \emph{«~non-free~»} («~non
libre~»). Les paquets de ces sections ne font pas partie du système Debian,
bien qu'ils aient été configurés afin d'être utilisés avec lui. Nous
encourageons les fabricants de CD à lire les licences de ces paquets afin de
déterminer s'ils peuvent les distribuer. Ainsi, bien que les travaux non libres
ne fassent pas partie de Debian, nous prenons en compte leur utilisation et
fournissons donc l'infrastructure nécessaire (à l'image de notre système de
suivi des bogues et de nos listes de diffusion).
}

% ****** DFSG ******

\def\DFSGT{Les principes du logiciel libre selon Debian (DFSG)}

\def\DFSGa{Redistribution libre et gratuite.}

\def\DFSGat{
La licence d'un composant de Debian ne doit pas empêcher quiconque de vendre
ou de donner le logiciel sous forme de composant d'un ensemble (distribution)
constitué de programmes provenant de différentes sources. La licence ne doit
en ce cas requérir ni redevance ni rétribution.
}

\def\DFSGb{Code source.}

\def\DFSGbt{
Le programme doit inclure le code source et sa diffusion sous forme de code
source comme de programme compilé doit être autorisée.
}

\def\DFSGe{Applications dérivées.}

\def\DFSGet{
La licence doit autoriser les modifications et les applications dérivées ainsi
que leur distribution sous les mêmes termes que ceux de la licence du logiciel
original.
}

\def\DFSGf{Intégrité du code source de l'auteur.}

\def\DFSGft{
La licence peut défendre de distribuer le code source modifié
\textbf{seulement} si elle autorise la distribution avec le code source de
fichiers correctifs destinés à modifier le programme au moment de sa
construction. La licence doit autoriser explicitement la distribution de
logiciels créés à partir de code source modifié. Elle peut exiger que les
applications dérivées portent un nom ou un numéro de version différent de ceux
du logiciel original (c'est un compromis~; \emph{le groupe Debian
encourage tous les auteurs à ne restreindre en aucune manière les modifications
des fichiers, source ou binaire}).
}

\def\DFSGg{Aucune discrimination de personne ou de groupe.}

\def\DFSGgt{
La licence ne doit discriminer aucune personne ou groupe de personnes.
}

\def\DFSGh{Aucune discrimination de champ d'application.}

\def\DFSGht{
La licence ne doit pas défendre d'utiliser le logiciel dans un champ
d'application particulier. Par exemple, elle ne doit pas défendre l'utilisation
du logiciel dans une entreprise ou pour la recherche génétique.
}

\def\DFSGi{Distribution de licence}

\def\DFSGit{
Les droits attachés au programme doivent s'appliquer à tous ceux à qui il est
distribué sans obligation pour aucune de ces parties de se conformer à une
autre licence.
}

\def\DFSGj{La licence ne doit pas être spécifique à Debian}

\def\DFSGjt{
Les droits attachés au programme ne doivent pas dépendre du fait de son
intégration au système Debian. Si le programme est extrait de Debian et utilisé
et distribué sans Debian mais sous les termes de sa propre licence, tous les
destinataires doivent jouir des même droits que ceux accordés lorsqu'il se
trouve au sein du système Debian.
}

\def\DFSGk{La licence ne doit pas contaminer d'autres logiciels}

\def\DFSGkt{
La licence ne doit pas placer de restriction sur d'autres logiciels distribués
avec le logiciel. Elle ne doit par exemple pas exiger que tous les autres
programmes distribués sur le même support soient des logiciels libres.
}

\def\DFSGl{Exemples de licence}

\def\DFSGlt{
Les licences «~GPL~», «~BSD~» et «~Artistic~» sont des exemples de licences que
nous considérons \emph{«~libres~»}.
}

\def\Outro{
L'idée de rédiger nos «~principes du logiciel libre~» fut suggérée par Ean
Schuessler. Bruce Perens écrivit la première ébauche de ce document et la
perfectionna d'après les commentaires des développeurs de Debian recueillis à
l'occasion d'une conférence tenue par courriels interposés pendant tout le mois
de juin~1997. Le document résultant fut ensuite accepté comme faisant
partie intégrante de la charte du projet Debian.

Plus tard, Bruce Perens retira toute référence au projet Debian des DFSG pour
en faire la «~Définition de l'Open Source~».

D'autres organisations peuvent utiliser ce document. Si vous le faites,
veuillez faire référence au projet Debian.
}

\def\References{
Le contrat social Debian est publié à l'adresse
\texttt{https://www.debian.org/social\_contract}~; vos trouverez plus
d'informations sur Debian à la page \texttt{https://www.debian.org/intro/about}
et des images d'installation peuvent être téléchargées à partir de
\texttt{https://www.debian.org/distrib/}
}
