\selectlanguage{italian}

\def\Title{Il Contratto Sociale Debian}

\def\Version{
La versione 1.1 approvata il 26 aprile 2004 che sostituisce la versione 1.0
approvata il 5 luglio 1997.
}

\def\Intro{
Debian, i produttori del sistema Debian, ha creato il \textbf{Contratto Sociale
Debian (Debian Social Contract)}. Le Linee Guida Debian per il Software Libero
(Debian Free Software Guidelines - DFSG) che sono parte del contratto,
inizialmente pensate come un insieme di punti che noi tutti sottoscriviamo,
sono state adottate dalla comunità del software libero come base per l'Open
Source Definition.
}

% ****** Social Contract ******

\def\SCT{\emph{“Contratto Sociale”} con la Comunità Free Software}

\def\SCa{Debian rimarrà libera al 100\%}

\def\SCat{Forniamo le linee guida che usiamo per determinare se un'opera sia
\emph{“libera”} nel documento intitolato \emph{“The Debian Free Software
Guidelines”}. Promettiamo che il sistema Debian e tutte le sue componenti
rimarranno liberi in accordo con le citate linee guida. Supporteremo le persone
che creino o usino in Debian opere sia libere che non libere. Non renderemo mai
il sistema dipendente da un componente non libero.}

\def\SCb{Renderemo alla Comunità Free Software}

\def\SCbt{Quando scriviamo nuovi componenti del sistema Debian, li rilasceremo
con una licenza che rispetti le Debian Free Software Guidelines. Realizzeremo
il sistema migliore che potremo, cosicché le opere libere siano usate e
distribuite il più possibile. Comunicheremo cose come bug fix, migliorie e
richieste degli utenti agli autori \emph{upstream} delle opere incluse nel
nostro sistema.}

\def\SCc{Non nasconderemo i problemi}

\def\SCct{Manterremo sempre il nostro intero bug report database aperto alla
pubblica lettura. I rapporti che le persone invieranno online saranno
prontamente resi visibili a tutti.}

\def\SCd{Le nostre priorità sono gli utenti ed il software libero}

\def\SCdt{Ci faremo guidare dai bisogni dei nostri utenti e della comunità del
software libero. Metteremo al primo posto i loro interessi. Supporteremo le
necessità dei nostri utenti di operare in molti diversi tipi di ambienti di
calcolo. Non ci opporremo alle opere non libere che siano state pensate per
l'uso in sistemi Debian e non richiederemo compensi a chi crea o usa queste
opere. Permetteremo ad altri di creare distribuzioni contenenti sia il sistema
Debian che altre opere, senza richiedere compensi. Per raggiungere questi
scopi, forniremo un sistema integrato di materiali di alta qualità senza alcuna
restrizione legale che limiti qualsiasi uso del sistema.}

\def\SCe{Opere che non rispettano i nostri standard free software}

\def\SCet{Ci rendiamo conto che alcuni dei nostri utenti richiedono di usare
opere non conformi alle Debian Free Software Guidelines. Abbiamo creato le aree
\emph{“contrib”} e \emph{“non-free”} nel nostro archivio per queste opere. I
pacchetti in queste aree non fanno parte del sistema Debian, sebbene siano
stati configurati per l'uso con Debian. Invitiamo i realizzatori di CD a
leggere le licenze dei pacchetti in queste aree per determinare se possono
distribuire i pacchetti sui loro CD. Inoltre, anche se le opere non libere non
fanno parte di Debian, supporteremo il loro uso e forniremo infrastrutture per
i pacchetti non liberi (come il nostro bug tracking system e le mailing list).}

% ****** DFSG ******

\def\DFSGT{Le Linee Guida Debian per il Software Libero}

\def\DFSGa{Libera ridistribuzione}

\def\DFSGat{La licenza di un componente Debian non può porre restrizioni a
nessuno per la vendita o la cessione del software come componente di una
distribuzione software aggregata di programmi proveniente da fonti diverse. La
licenza non può richiedere royalty o altri pagamenti per la vendita.}

\def\DFSGb{Codice sorgente}

\def\DFSGbt{Il programma deve includere il codice sorgente e deve permettere la
distribuzione sia come codice sorgente che in forma compilata.}

\def\DFSGe{Lavori derivati}

\def\DFSGet{La licenza deve permettere modifiche e lavori derivati e deve
permettere la loro distribuzione con i medesimi termini della licenza del
software originale.}

\def\DFSGf{Integrità del codice sorgente dell'autore}

\def\DFSGft{La licenza può porre restrizioni sulla distribuzione di codice
sorgente modificato \textbf{solo} se permette la distribuzione di \emph{“file
patch”} insieme al codice sorgente con lo scopo di modificare il programma
durante la compilazione. La licenza deve esplicitamente permettere la
distribuzione di software compilato con codice sorgente modificato. La licenza
può richiedere che i lavori derivati abbiano un nome o un numero di versione
diversi da quelli del software originali. (\emph{Questo è un compromesso. Il
gruppo Debian invita tutti gli autori a non impedire che file, sorgenti o
binari possano essere modificati.})}

\def\DFSGg{Nessuna discriminazione di persone o gruppi}

\def\DFSGgt{La licenza non può discriminare nessuna persona o gruppo di
persone.}

\def\DFSGh{Nessuna discriminazione nei campi di impiego}

\def\DFSGht{La licenza non può porre restrizioni all'utilizzo del programma in
uno specifico campo di impiego. Per esempio, non può porre restrizioni all'uso
commerciale o nella ricerca genetica.}

\def\DFSGi{Distribuzione della licenza}

\def\DFSGit{I diritti applicati al programma devono essere applicabili a
chiunque riceva il programma senza il bisogno di utilizzare licenze addizionali
di terze parti.}

\def\DFSGj{La licenza non può essere specifica per Debian}

\def\DFSGjt{I diritti applicati al programma non possono dipendere dal fatto
che esso sia parte di un sistema Debian. Se il programma è estratto da Debian e
usato o distribuito senza Debian ma ottemperando ai termini della licenza,
tutte le parti alle quali il programma è ridistribuito dovrebbero avere gli
stessi diritti di coloro che lo ricevono con il sistema Debian.}

\def\DFSGk{La licenza non deve contaminare altro software}

\def\DFSGkt{La licenza non può porre restrizioni ad altro software che sia
distribuito insieme al software concesso in licenza. Per esempio, la licenza
non può richiedere che tutti gli altri programmi distribuiti con lo stesso
supporto debbano essere software libero.}

\def\DFSGl{Esempi di licenze}

\def\DFSGlt{Le licenze GPL, BSD e Artistic sono esempi di licenze che
consideriamo libere.}

\def\Outro{I concetti enunciati nel nostro contratto sociale con la comunità
del software libero furono proposti da Ean Schuessler. Questo documento fu
abbozzato da Bruce Perens, rifinito da altri sviluppatori Debian durante una
conferenza via posta elettronica durata un mese (Giugno 1997) ed infine
approvata come pubblica linea di condotta del Progetto Debian.

Bruce Perens in seguito ha rimosso i riferimenti specifici a Debian dalle Linee
Guida Debian per il Free Software per creare The Open Source Definition.

Altre organizzazioni possono derivare da questo documento. Per favore, se lo si
fa, se ne dia credito al progetto Debian.}

\def\References{Il Contratto Sociale Debian è pubblicato all'indirizzo
\texttt{https://www.debian.org/social\_contract.it.html}; ulteriori
informazioni su Debian si trovano su
\texttt{https://www.debian.org/intro/about.it.html} e le immagini per
l'installazione sono scaricabili da \texttt{https://www.debian.org/distrib/}.}
