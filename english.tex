\selectlanguage{english}

\def\Universal{The Universal Operating System}

\def\WhatIsDebianCaption{What is Debian GNU/Linux?}

\def\WhatIsDebian{
%
Debian is a free operating system (OS) for your computer. An operating
system is the set of basic programs and utilities that make your
computer run. At the core of every operating system is the kernel.  It
is the most fundamental program on the computer: it does all the basic
housekeeping and lets you start other programs. Debian is currently
based on the Linux kernel and includes more than 9,500 packages of
utilities and applications. Almost 1,000 developers are working hard
to maintain Debian's high quality.
%
}

\def\FreedomCaption{Freedom}

\def\Freedom{
%
Debian is comprised entirely of Free Software.  By Free Software, we
don't just mean zero cost, but also the Freedom to use it how and where you
want, share it with your friends, read and modify the source code, and
distribute those changes to other people.  This means Debian can be
used with no limitations - even in commercial environments!  Debian is
the largest collection of ready-to-install Free Software on the Internet.
%
}

\def\CommunityCaption{Community}

\def\Community{
%
The Debian project is a 100\% volunteer effort aiming at producing
a world class Open Source operating system. There are currently about
1,000 people from around the world developing the Debian
system, each with roles ranging from package development through
quality assurance, security, policy, and strategy. The Debian project
is engaged to the principles of software freedom and openness. Its
commitment is plainly stated in the Debian Social Contract published
at
\boxurl{http://www.debian.org/social_contract}. The Debian Free Software
Guidelines describe the criteria that licenses for software included
in the Debian operating system must meet. The Open Source Definition
is a derived work of the Debian Free Software Guidelines. 
%
}

\def\ContinuityCaption{Continuity}

\def\Continuity{
%
The Debian packaging system permits a seamless transition to newer
program versions without the requirement to begin a new installation
from scratch, and it won't delete your old configuration.
Dependencies between programs are handled automatically:
If a package which you want to install
requires another package, the installer takes care of it.
You can install and upgrade using disks,
CD-ROMs, or over a network connection.
%
}

% Fixme: there should probably be a subheading "Continuity"
% and this text should probably go under some subheading,
% but do they really belong together? How about adding
% "ease of maintenance" and plug the packaging system
% in that context instead? (I'd avoid the monstrous word
% "maintainability" ... and it's hard to translate elegantly)


\def\StabilityCaption{Stability}

\def\Stability{
%
Debian has no commercial pressure and will not release a new and
possibly unstable version just because the market requires that. The
Debian maintainers always test the system thoroughly and attempt to
remove all known bugs before releasing a new version.
%
}

\def\PortabilityCaption{Portability}

\def\Portability{
%
Debian is available and runs equally well on the following
architectures: Alpha, ARM, HP PA-RISC, IBM S/390, Intel x86, Intel
IA-64, Motorola 68k, MIPS/MIPSel, PowerPC, SPARC.
%
}

\def\IncludedCaption{Included with Debian GNU/Linux}

\def\Included{
%
Actually, the complete Debian GNU/Linux distribution fits barely on 6 CDs
(architecture-dependent precompiled binaries, even more CDs with source). Inside
you'll find:
%
}

\def\Utilities{
%
the full set of GNU utilities, editors (emacs, vi,~\ldots), network
clients (telnet, ftp, finger,~\ldots), web browsers, privacy tools
(gpg, ssh,~\ldots), email clients, and every little tool you can think
of
%
}

\def\Networking{
%
full set of network protocols (PPP, TCP/IP, Apple\tm\ EtherTalk,
Windows\tm\ SMB, Novell\tm,~\ldots)
%
}

\def\Programming{
%
development tools for the major programming languages (and some of the
more obscure ones as well) like: C, C++, Objective-C, Java, Python,
Perl, Smalltalk, Lisp, Scheme, Haskell, Ada, and more
%
}

\def\Windowsystem{
%
the X11 Window System, complete with dozens of window managers and the
two leading desktops: Gnome and KDE
%
}

\def\Documents{
%
the \TeX/\LaTeX\ document preparation system, PostScript\tm\ and Type1
fonts and tools, the Ghostscript PostScript\tm\ interpreter, and a
complete XML/SGML/HTML development environment
%
}

\def\Graphics{
%
GIMP, the GNU Image Manipulation Program (a free alternative to
Photoshop\tm)
%
}

\def\Office{
%
a complete set of office applications: WYSIWYG editors, calendars,
spreadsheets, databases, etc.
%
}

\def\Databases{
%
relational databases, like PostgreSQL, MySQL and
development tools (application servers, server side scripting
languages)
%
}

\def\KnowMoreCaption{Want to know more?}

\def\KnowMore{
%
Simply point your browser at
\boxurl{http://www.debian.org/}. If you need any information or help you
can join the IRC channel \#debian on \boxurl{irc.debian.org}, or
one of the Debian mailing lists. See
\boxurl{http://www.debian.org/MailingLists/subscribe} for instructions.
%
}

\def\Install{
%
If you want to install Debian GNU/Linux, you can download the install
floppies from \boxurl{ftp://ftp.debian.org/} and then go for a network
install, or order some Debian CDs. Debian does not sell CDs itself but
provides Official CD Images that numerous vendors print and sell. 
For details about the Official CD Images simply go to
\boxurl{http://www.debian.org/CD/}.
%
}

\def\MadeWith{
%
This flyer was made using \LaTeX\ and a Debian system.
%
}

\def\SponsoredBy{
%
Printing of this flyer was sponsored by
%
}

\def\SponsorURL{
%
http://www.credativ.de/
%
}

% EPS file in sponsors/ subdirectory
\def\SponsorLogo{credativ}
