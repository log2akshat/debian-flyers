%   Copyright (c) 2000,2002 Software in the Public Interest, Inc.
%
%   This program is free software; you can redistribute it and/or modify
%   it under the terms of the GNU General Public License as published by
%   the Free Software Foundation; version 2 dated June, 1991.
%
%   This program is distributed in the hope that it will be useful,
%   but WITHOUT ANY WARRANTY; without even the implied warranty of
%   MERCHANTABILITY or FITNESS FOR A PARTICULAR PURPOSE.  See the
%   GNU General Public License for more details.
%
%   You should have received a copy of the GNU General Public License
%   along with this program;  if not, write to the Free Software
%   Foundation, Inc., 59 Temple Place - Suite 330, Boston, MA 02111, USA.

%   translation-check translation="1.41"

% this is version Mon, 20 Sep 2004 22:22:12 +0200

\selectlanguage{german}

\hyphenation{Debian}

\def\Universal{Das universelle Betriebssystem}

\def\WhatIsDebianCaption{Was ist Debian GNU/Linux?}

\def\WhatIsDebian{
%
Debian ist ein freies Betriebssystem f�r Ihren Computer. Debian
basiert gegenw�rtig auf dem Linux-Kernel und enth�lt mehr als 29~000
Pakete mit Programmen und Anwendungen -- weit mehr
Software, als f�r jedes andere freie Betriebssystem erh�ltlich ist.
9 Architek\-turen werden unterst�tzt, darunter 32-bit und 64-bit PCs,
sowie technische Previews zweier neuer Portierungen auf den Kernel des
FreeBSD-Projekts. 
Seit 1993 arbeiten freiwillige Entwickler (etwa 1~000) daran, Debians hohe
Qualit�t aufrecht zu erhalten.
%
}

\def\FreedomCaption{Freiheit}

\def\Freedom{
%
Nach dem Debian-Gesellschaftsvertrag be\-steht Debian g�nzlich
aus Freier Software.  Mit "`Freier Software"' meinen wir nicht nur
"`null Kosten"', sondern auch die Freiheit, sie zu benutzen, wie und wo
Sie wollen, sie mit Ihren Freunden zu teilen, den Quelltext zu lesen
und zu ver�ndern und diese �nderungen an andere Leute zu verteilen.
%
}

\def\ContinuityCaption{Kontinuit�t}

\def\Continuity{
%
Das Paketverwaltungssystem von Debian erlaubt einen nahtlosen
�bergang zu neuen Programmversionen, ohne dass eine komplette
Neuinstallation notwendig ist; Ihre alten Konfigurations\-dateien
bleiben erhalten.  Die Abh�ngigkeiten zwischen den
Programmen werden automatisch aufgel�st: wenn ein Paket, welches Sie
installieren wollen, ein anderes Paket ben�tigt, k�mmert sich das
Installationsprogramm darum.  Sie k�nnen von Disketten, CD-ROMs, USB-Sticks oder
�ber eine Netzwerk\-verbindung installieren und Upgrades durchf�hren.
%
}

\def\SecurityCaption{Sicherheit}

\def\Security{
Debian ist per Voreinstellung sicher; au�er\-dem leicht anpassbar bei wirklich
extremen Sicher\-heits\-anforderungen.  Es ist leicht, ein Debian-System
mit nur sehr wenig Software zu installieren.  Wenn ein
Sicherheits-Problem gefunden wurde, ver�ffentlicht das Debian-Sicherheitsteam
Pakete mit r�ckportierten
Sicherheits-Aktualisierungen, meistens innerhalb von 48
Stunden.  Mit der Hilfe des Debian-Paket\-verwaltungs\-systems ist es
daher auch f�r einen unerfahrenen Administrator leicht, die Sicherheit auf
dem neuesten Stand zu halten.  Dieser Dienst ist f�r jeden kostenlos
verf�gbar.
%
}

\def\QualityCaption{Qualit�t}

\def\Quality{
%
Aufgrund des pers�nlichen Engagements der Debian-Entwickler ist die
Paket-Qualit�t beispiellos.  Das Debian-Richtlinien-Handbuch (Debian
Policy Manual), welches genau beschreibt, wie Pakete sich verhalten
und mit dem System interagieren sollten, hilft weiter dabei, diesen
Standard aufrecht zu erhalten.  Ein Paket, welches die Richtlinien
verletzt, wird nicht in die offiziell freigegebene stabile Version einbezogen.
Eine weitere Hilfe bei der Qualit�tssicherung ist die
Debian-Fehler\-datenbank (Debian Bug Tracking System).  Debian versteckt
keine Probleme.  Debian unterliegt keinem kommerziellen Druck und wird
keine neue und m�glicherweise instabile Version ver�ffentlichen, nur
weil der Markt es erfordert.
%
}
\def\IncludedCaption{Enthalten in Debian GNU/Linux}

\def\Included{
%
Die vollst�ndige Debian-GNU/Linux-Distribution besteht aus
mehr als 7 DVDs, 44 CDs oder 2 BDs.  Darin finden Sie:
%
}

\def\Utilities{
%
alle GNU-Hilfsprogramme, Editoren (Emacs,
vi,~\ldots), Netzwerk-Clients (Chat, File-Sharing,~\ldots), Webbrowser
(z.\,B.~Mozilla), Werkzeuge f�r die Privatsph�re
(gpg, ssh,~\ldots), E-Mail-Clients und jedes kleine Hilfsmittel, das Sie
sich vorstellen k�nnen
%
}

\def\Networking{
%
einen vollst�ndigen Satz von Netzwerkprotokollen (PPP, IPv4, IPv6,
SMB-Netzwerkumgebung,~\ldots)
%
}

\def\Programming{
%
Entwicklungswerkzeuge f�r die gro�en Programmiersprachen wie C, C++,
Objective-C, Java, Python, Perl und auch einige der weniger bekannten
%
}

\def\Windowsystem{
%
das grafische X11 Window System, vervollst�ndigt durch Dutzende von
Fenstermanagern und die drei f�hrenden Arbeits\-umgebungen: GNOME,
KDE ``Plasma Desktop and Applications'' und LXDE
%
}

\def\Documents{
%
das \TeX/\LaTeX-Satzsystem, PostScript\tm\ u.~Type1-Schrift\-arten
und -Werkzeuge, den PostScript\tm-Interpreter Ghostscript und eine
komplette XML/SGML/HTML-Entwicklungsumgebung
%
}

\def\Graphics{
%
GIMP, das GNU-Bildbearbeitungsprogramm (eine freie Alternative zu
Photoshop\tm)
%
}

\def\Office{
%
einen vollst�ndigen Satz von B�ro-Anwendungen, darunter die
Produktivit�ts-Suite OpenOffice.org, Gnumeric und andere
Tabellenkalkulationen, WYSIWYG-Editoren, Kalender
%
}

\def\Databases{
%
relationale Datenbanken wie PostgreSQL und MySQL sowie
Entwicklungswerkzeuge (Anwendungs-Server, serverseitige
Skriptsprachen)
%
}

\def\KnowMoreCaption{M�chten Sie mehr wissen?}

\def\KnowMore{
%
Dann besuchen Sie die Webseite \boxurl{http://www.debian.org/}. F�r
mehr Informationen oder Hilfe besuchen Sie den IRC-Kanal \#debian
auf \boxurl{irc.debian.org} oder abonnieren Sie eine der
Debian-Mailing\-listen unter
\boxurl{http://www.debian.org/MailingLists/subscribe}. Au�erdem
gibt es eine Menge inoffizieller Benutzer-Foren.
%
}

\def\Install{
%
Zur Installation von Debian GNU/Linux k�nnen Sie die
Installations CDs von \boxurl{ftp://ftp.de.debian.org/}
f�r eine Netzwerkinstallation herunterladen oder
Debian DVDs/CDs bestellen (oder brennen).  Debian verkauft keine
DVDs/CDs, stellt aber offizielle Images bereit, die zahlreiche
H�ndler vertreiben.  Details �ber die offiziellen
DVD/CD-Images gibt es unter
\boxurl{http://www.debian.org/CD/}.
%
}

\def\MadeWith{
%
Erstellt am \today\\
mit \LaTeX\ auf einem Debian-System.
}

\def\SponsoredBy{Der Druck dieses Flyers wurde gesponsert von}

% URL of the sponsor
%\def\SponsorURL{http://www.credativ.de/}
\def\SponsorURL{http://www.ffis.de/}


% EPS file in sponsors/ subdirectory
%\def\SponsorLogo{credativ}
\def\SponsorLogo{ffis-logo-color}

