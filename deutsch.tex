%   Copyright (c) 2000,2002 Software in the Public Interest, Inc.
%
%   This program is free software; you can redistribute it and/or modify
%   it under the terms of the GNU General Public License as published by
%   the Free Software Foundation; version 2 dated June, 1991.
%
%   This program is distributed in the hope that it will be useful,
%   but WITHOUT ANY WARRANTY; without even the implied warranty of
%   MERCHANTABILITY or FITNESS FOR A PARTICULAR PURPOSE.  See the
%   GNU General Public License for more details.
%
%   You should have received a copy of the GNU General Public License
%   along with this program;  if not, write to the Free Software
%   Foundation, Inc., 59 Temple Place - Suite 330, Boston, MA 02111, USA.
\selectlanguage{german} 

% FIXME drop obsolete rcsInfo
\def\MyRCS{\rcsInfo $Id: english.tex,v 1.46 2017/02/16 04:48:33 joostvb Exp $} 

\def\Universal{Das universelle Betriebssystem} 

\def\WhatIsDebianCaption{Was ist Debian GNU/Linux?} 

\def\WhatIsDebian{ 

%
Debian ist ein freies Betriebssystem für Ihren Computer. Debian basiert gegenwärtig auf dem Linux-Kernel und enthält
mehr als 59~000 Pakete mit Programmen und Anwendungen. --weit mehr als für jedes andere freie Betriebssystem.. 10 Architek\-turen werden unterstützt, darunter 32- und 64-bit PCs. Seit 1993 arbeiten etwa 1~000 freiwillige Ent\-wickler daran,
Debians hohe Qualität aufrecht zu erhalten. 

%
} 

\def\FreedomCaption{Freiheit} 

\def\Freedom{ 

%
Der Gesellschaftsvertrag von Debian fordert ein,
daß es vollkommen aus Freier Software besteht.. Mit "`Freier Software"' meinen wir nicht nur "`kostenfrei"',
sondern vor allem die Freiheiten, sie uneingeschränkt zu verwenden,
mit Freunden zu teilen, den Quelltext zu lesen können und auch
zu verändern, und diese Änderungen ebenfalls verteilen zu dürfen. 

%
} 

\def\ContinuityCaption{Kontinuität} 

\def\Continuity{ 

%
Das Paketverwaltungssystem von Debian erlaubt einen nahtlosen Übergang
zu neuen Programmversionen, ohne dass eine komplette Neuinstallation
notwendig ist; auch Ihre alten Konfigurations\-dateien bleiben dabei erhalten. Die Abhängigkeiten zwischen den Programmen werden automatisch aufgelöst:
wenn ein Paket ein anderes Paket benötigt, so kümmert
sich das Installationsprogramm um die Abhängigkeiten.. Sie können Programmpakete von den üblichen Datenträgern
wie CDs, DVDs, USB-Sticks oder auch über eine Netzwerk\-verbindung (Kabel, W-LAN) installieren und aktualisieren. Das, was am besten zu Ihnen passt. 

%
} 

\def\SecurityCaption{Sicherheit} 

\def\Security{ 

%
Debian ist in seinen Voreinstellungen sehr sicher.
Es ist außer\-dem leicht anpassbar und hat
wirklich extreme Sicher\-heits\-anforderungen.. Es ist leicht, ein Debian-System mit nur
sehr wenig Software zu installieren. Wenn ein Sicherheits-Problem gefunden wurde,
veröffentlicht das Sicherheitsteam die Pakete
mit rückportierten Sicherheits-Aktualisierungen,
meist schon innerhalb von 48 Stunden. Mit der Hilfe des Debian-Paket\-verwaltungs\-systems ist es
daher auch für einen unerfahrenen Administrator leicht,
die Sicherheit auf dem neuesten Stand zu halten.. Dieser Dienst ist für jeden kostenlos verfügbar.. 

%
} 

\def\QualityCaption{Qualität} 

\def\Quality{ 

%
Danke an das persönlichen Engagements der Debian-Entwickler
ist die Qualität der Pakete beispiellos.
Das Debian-Richtlinien-Handbuch (Debian Policy Manual) unterstützt, diesen Standard aufrechtzuhalten.
,Die Aufrechterhaltung dieses Standards wird außerdem durch das Debian Policy Manual unterstützt, Hier wird genau beschrieben, wie sich Pakete verhalten und mit dem System interagieren sollten. Ein Paket, welches die Richtlinien verletzt,
wird nicht in die offiziell freigegebene stabile Version einbezogen. Eine weitere Hilfe bei der Qualitätssicherung ist die öffentliche
Debian-Fehler\-datenbank (Debian Bug Tracking System).. Debian versteckt keine Probleme.. Debian unterliegt keinem kommerziellen Druck und wird keine neue
und möglicherweise instabile Version veröffentlichen,
nur weil der Markt es erfordert. 

%
} 

\def\IncludedCaption{Enthalten in Debian GNU/Linux} 

\def\Included{ 

%
Die vollständige Debian-GNU/Linux-Distribution besteht aus
mehr als 12 DVDs, 74 CDs oder 3 BDs.. Diese Datenträger enthalten nur die architekturabhängigen vorkompilierten Binärdateien. Die Quellen sind auf separaten Medien erhältlich Darin finden Sie: 

%
} 

\def\Utilities{ 

%
alle GNU-Dienstprogramme, Editoren (Emacs, vi, \ldots), Netzwerk-Clients (Chat, Filesharing, \ldots), Webbrowser (einschließlich Chromium sowie Firefox und andere Mozilla-Produkte), Tools zum Schutz der Privatsphäre (gpg, ssh, \ldots), E-Mail-Clients und viele andere kleine Werkzeuge, die Ihnen einfallen. 

%
} 

\def\Networking{ 

%
einen vollständigen Satz von Netzwerkprotokollen
(PPP, IPv4, IPv6, SMB-Netzwerkumgebung,~\ldots); 

%
} 

\def\Programming{ 

%
Entwicklungswerkzeuge für die wichtigsten Programmiersprachen wie C (über GCC und LLVM/Clang), Java, Python, PHP, Ruby, Perl5 und Raku sowie für viele der weniger bekannten Sprachen;; 

%
} 

\def\Windowsystem{ 

%
das grafische X Window System,
vervollständigt durch Dutzende von Fenstermanagern
und die sechs führenden Arbeits\-umgebungen:
GNOME, MATE, KDE, Cinnamon, XFCE und LXDE. 

%
} 

\def\Documents{ 

%
das \TeX/\LaTeX\ Dokumentenvorbereitungssystem, PostScript\tm\ und Type1-Schriften und -Werkzeuge, den Ghostscript PostScript\tm\ Interpreter und eine vollständige XML/SGML/HTML-Entwicklungsumgebung. 

%
} 

\def\Graphics{ 

%
GIMP, das GNU-Bildbearbeitungsprogramm
(eine freie Alternative zu Photoshop\tm); 

%
} 

\def\Office{ 

%
einen vollständigen Satz von Büro-Anwendungen,
darunter die Produktivitäts-Suite LibreOffice,
Gnumeric und andere Tabellenkalkulationen,
WYSIWYG-Editoren, Kalender 

%
} 

\def\Databases{ 

%
relationale Datenbanken wie PostgreSQL und MariaDB,
sowie Entwicklungswerkzeuge
(Anwendungs-Server, serverseitige Skriptsprachen). 

%
} 

\def\KnowMoreCaption{Möchten Sie mehr wissen?} 

\def\KnowMore{ 

%
Dann besuchen Sie die Webseite \boxurl{https://www.debian.org}. Für weitere Informationen oder Hilfe können Sie dem IRC-Kanal \#debian.de auf \boxurl{irc.debian.org}, oder einer der Debian-Mailinglisten beitreten.. Lesen Sie \boxurl{https://www.debian.org/MailingLists/subscribe} für Hinweise. Außerdem gibt es eine Menge inoffizieller Benutzer-Foren.. 

%
} 

\def\Install{ 

%
Zur Installation von Debian GNU/Linux können Sie ein Installmedium von \boxurl{https://www.debian.org/distrib} herunterladen, eine Netzwerkinstallation starten oder einen Satz an DVDs/CDs kaufen. Debian verkauft keine DVDs/CDs,
stellt aber offizielle Images bereit,
die zahlreiche Händler vertreiben. Debian also offers cloud images, for OpenStack, EC2, Azure and other cloud providers. Details über die offiziellen
DVD/CD-Images gibt es unter \boxurl{https://www.debian.org/CD}. 

%
} 

\def\MadeWith{ 

%
% \today is from rcsInfo
Dieses Flugblatt wurde erstellt am \today\\ mit \LaTeX\ und einem Debian-System. 

%
} 

%\def\SponsoredBy{Printing of this flyer was sponsored by}
\def\SponsoredBy{Der Druck dieses Flyers wurde gesponsert von.} 

% for simple text, use \SponsorDonations in layout.tex
% URL of the sponsor
% \def\SponsorURL{http://www.credativ.de/}
\def\SponsorURL{http://www.powercraft.nl/} 

% \def\SponsorURL{http://www.ffis.de/}
% EPS file in sponsors/ subdirectory
% \def\SponsorLogo{credativ}
\def\SponsorLogo{powercraft} 

% \def\SponsorLogo{ffis-logo-color}
