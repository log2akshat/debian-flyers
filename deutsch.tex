% Contributed by Martin Banck <mbanck@debian.org>
%
\selectlanguage{german}

\def\Universal{Das universelle Betriebssystem}

\def\WhatIsDebianCaption{Was ist Debian GNU/Linux?}

\def\WhatIsDebian{
%
Debian ist ein freies Betriebssystem f�r Ihren Computer. Ein
Betriebssystem besteht aus Programmen und Werkzeugen, mit denen Sie
Ihren Computer betreiben k�nnen. Die Grundlage eines Betriebssystems
bildet der Kernel. Der Kernel ist der wichtigste Teil eines
Betriebssystems. Er verwaltet den Computer und erlaubt es, andere
Programme zu starten. Aktuell basiert Debian auf dem Linux Kernel und
es enth�lt zur Zeit �ber 9\,500 Pakete, die von ca.~1\,000 Entwicklern
betreut werden.
%
}

\def\FreedomCaption{Freiheit}

\def\Freedom{
%
Debian besteht komplett aus freier Software. Dies erlaubt einen
uneingeschr�nkten Einsatz von Debian, sogar in kommerziellen
Umgebungen. Der Quelltext aller Programme ist frei erh�ltlich und ist
in Debian enthalten. Debian bietet die gr��te Sammlung an freier
Software, die vorkompiliert und startbereit installiert werden kann.
%
}

\def\CommunityCaption{Community}

\def\Community{
%
Das Debian-Projekt besteht zu 100\% aus freiwilliger Arbeit und hat
sich zum Ziel gesetzt, ein einzigartiges Open-Source-Betriebssystem zu
erstellen.  Derzeit arbeiten weltweit ca.~1\,000 Leute an Debian,
jeder in verschiedenen Bereichen, von der Paketentwicklung, �ber die
Qualit�tssicherung, bis zu Sicherheit, Richtlinien und Strategie.  Das
Debian-Projekt hat sich den Prinzipien der freien Software und der
Offenheit verschrieben.  Diese Grunds�tze sind im Debian Social
Contract unter \boxurl{http://www.debian.org/social_contract}
festgehalten. Die Definition von Open-Source ist von den Debian Free
Software Guidelines abgeleitet worden. Diese beschreiben die
Kriterien, unter welchen Lizenzbedingungen Software als frei gilt und
in Debian aufgenommen werden kann.
%
}

\def\ContinuityCaption{Kontinuit�t}

\def\Continuity{
%
Das Paket-Verwaltungssystem von Debian erlaubt einen nahtlosen
�bergang zu neuen Programmversionen ohne eine Neuinstallation; alte
Konfigurationsdaten werden nicht �berschrieben. Die Abh�ngigkeiten
zwischen den Programmen werden automatisch aufgel�st. Sie k�nnen das
Sys\-tem von verschiedenen Speichermedien aktualisieren: Disketten,
Festplatte, CD-ROM oder direkt �ber das Internet.
%
}

\def\StabilityCaption{Stabilit�t}

\def\Stability{
%
Da Debian keine kommerziellen Interessen verfolgt, werden neue
Versionen nicht in kurzen Ab\-st�nden nach Erfordernissen des Marktes,
sondern in gr��eren Intervallen ver�ffentlicht, wenn das System stabil
genug ist.  Die Debian-Entwickler testen das System so gut wie nur
m�glich und versuchen auch alle ver�ffentlichungskritischen Fehler zu
beseitigen.
%
}

\def\PortabilityCaption{Portabilit�t}

\def\Portability{
%
Debian l�uft derzeit gleicherma�en gut auf den folgenden
Architekturen: Alpha, ARM, HP PA-RISC, IBM S/390, Intel IA-64, Intel
x86, MIPS/MIPSel, Motorola 68k, PowerPC, SPARC.
%
}

\def\IncludedCaption{Enthalten in Debian GNU/Linux}

\def\Included{
%
Die komplette Debian GNU/Linux Distribution passt auf 6 CDs
(architekturabh�ngige, vorkompilierte Pakete; weitere CDs mit dem
Quellcode). Darin finden Sie:
%
}

\def\Utilities{
%
Pakete mit GNU Werkzeugen, Editoren (emacs, vi,~\ldots), Netzwerk
Clients (telnet, ftp, finger,~\ldots), Web Browsern, Werkzeugen zum
Schutz der pers�nlichen Daten (gpg, ssh,~\ldots), E-Mail Programmen
und allen m�glichen weiteren Werkzeugen
%
}

\def\Networking{
%
Unterst�tzung aller wichtigen Netzwerkprotokolle (TCP/IP, PPP,
Apple\tm EtherTalk, Windows\tm SMB, Novell\tm,~\ldots)
%
}

\def\Programming{
%
Entwicklungswerkzeuge f�r alle g�ngigen und viele exotische
Programmiersprachen, wie: C, C++, Objective-C, Java, Python, Perl,
Smalltalk, Lisp, Scheme, Haskell, Ada und mehr
%
}

\def\Windowsystem{
%
das X11 Window System, erg�nzt durch Dutzende von Window-Managern und
die zwei meistbenutzten Desktop-Umgebungen, Gnome und KDE
%
}

\def\Documents{
%
das \TeX/\LaTeX-Dokumentationssystem, PostScript\tm- und
Type1-Schriftarten und -Werkzeuge, den Ghostscript PostScript\tm\
Interpreter und eine komplette XML/SGML/HTML Entwicklungsumgebung
%
}

\def\Graphics{
%
GIMP, das GNU Bildverarbeitungsprogramm (eine freie Alternative zu
Photoshop\tm)
%
}

\def\Office{
%
Office Anwendungen: WYSIWYG Editoren, Kalender, Tabellenkalkulationen,
Datenbanken, etc.
%
}

\def\Databases{
%
relationale Datenbanken, wie PostgreSQL, MySQL, und
Ent\-wicklungswerkzeuge (Applikations-Server, serverseitige
Skriptsprachen)
%
}

\def\KnowMoreCaption{Sie m�chten mehr wissen?}

\def\KnowMore{
%
Dann besuchen Sie die Debian-Webseite \boxurl{http://www.debian.org/},
abonnieren Sie eine oder mehrere der Debian-Mailinglisten unter
\boxurl{http://www.debian.org/MailingLists/subscribe} oder benutzen Sie die
IRC-Kan�le \#Debian.DE auf \boxurl{irc.fu-berlin.de} oder
\#debian auf \boxurl{irc.debian.org}.
%
}

\def\Install{
%
Wenn Sie Debian GNU/Linux installieren wollen, so k�nnen Sie die
Installationsdisketten von \boxurl{ftp://ftp.de.debian.org/}
herunterladen und dann mit einer Netzwerkinstallation oder mit einer
CD weitermachen. Debian selbst verkauft keine CDs,
aber viele Firmen vertreiben "`offizielle"' CDs von Debian.
N�heres zu den offiziellen Debian-CD-Images gibt es auf
\boxurl{http://www.debian.org/CD/}.
%
}

\def\MadeWith{
%
Erstellt mit \LaTeX\ und einem Debian-System.
%
}

\def\SponsoredBy{
%
Der Druck dieses Flyers wurde gesponsort von
%
}

% URL of the sponsor
\def\SponsorURL{http://www.credativ.de/}

% EPS file in sponsors/ subdirectory
\def\SponsorLogo{credativ}
