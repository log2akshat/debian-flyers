%   Copyright (c) 2002 Software in the Public Interest, Inc.
%
%   This program is free software; you can redistribute it and/or modify
%   it under the terms of the GNU General Public License as published by
%   the Free Software Foundation; version 2 dated June, 1991.
%
%   This program is distributed in the hope that it will be useful,
%   but WITHOUT ANY WARRANTY; without even the implied warranty of
%   MERCHANTABILITY or FITNESS FOR A PARTICULAR PURPOSE.  See the
%   GNU General Public License for more details.
%
%   You should have received a copy of the GNU General Public License
%   along with this program;  if not, write to the Free Software
%   Foundation, Inc., 59 Temple Place - Suite 330, Boston, MA 02111, USA.

%   translation-check translation="1.23"

% Contributed by Wouter Verhelst <wouter@debian.org>

% Updated by Bas Zoetekouw <bas@debian.org>

% Updated august - september 2004 by Joost van Baal <joostvb@debian.org>

% Thanks to Marianne Driessen ( http://www.driewerf.nl/ )
% for proofreading

% Imported stuff from r333 | wsl | 2004-09-19 13:58:16 +0200 (Sun, 19 Sep 2004),
% as found after running
%  $ svn co http://fruit.eu.org/svn/wsl/texmf/flyer
% : puntkomma's achter opsomming; trema's ipv umlauts; fixed formatting. Thanks
% to Wessel Dankers

\selectlanguage{dutch}

\def\MyRCS{\rcsInfo $Id: nederlands.tex,v 1.28 2017/02/16 04:48:33 joostvb Exp $}

\def\Universal{Het universele besturingssysteem}

\def\WhatIsDebianCaption{Wat is Debian GNU/Linux?}

\def\WhatIsDebian{Debian is een vrij besturingssysteem voor uw computer.  Op
dit moment is Debian gebaseerd op de Linux-kernel en bevat het meer dan 59.000
pakketten met allerlei programma's.  Dit is veel meer programmatuur dan er voor
andere vrije besturingssystemen beschikbaar is.  Debian ondersteunt tien
computerarchitecturen en varianten, waaronder 32- en 64-bits PC's.
Sinds 1993 zorgt een team van vrijwillige ontwikkelaars (op dit moment meer dan duizend) ervoor
dat de hoge kwaliteit van Debian gewaarborgd blijft.}

\def\FreedomCaption{Vrijheid}

\def\Freedom{Zoals vastgelegd in het Debian Sociale Contract, bestaat Debian
volledig uit Vrije Software. Dit betekent meer dan alleen dat de software
gratis is. Ook valt daaronder de vrijheid om de software voor elk doel te gebruiken, te delen
met uw vrienden en kennissen, de broncode te lezen en aan te passen en die
wijzigingen aan anderen door te geven.}

\def\ContinuityCaption{Continu"iteit}

\def\Continuity{Het packagingsysteem van Debian laat een probleemloze overgang
naar nieuwe versies van software toe, zonder dat u vanaf nul moet beginnen en
zonder dat uw oude configuratie overschreven zal worden. Onderlinge
afhankelijkheden tussen pakketten worden automatisch opgelost en u kunt
upgraden vanaf allerlei media: BluRay disks, DVD's, CD's, USB-sticks, en direct vanaf het
internet, net wat u het beste uitkomt.}

\def\SecurityCaption{Veiligheid}

\def\Security{Een standaard Debian-systeem is een veilig systeem, en kan
eenvoudig aangepast worden om aan extreme beveiligingseisen te voldoen.  Het is
gemakkelijk om een Debian-systeem in te richten met alleen de software die u
daadwerkelijk gebruikt.  Wanneer een beveiligingsprobleem is gevonden, stelt
Debians securityteam aangepaste pakketten beschikbaar
gericht op het oplossen van het beveiligingsprobleem.  Dit gebeurt over het
algemeen binnen 48 uur.  Met behulp van
Debians packagingsysteem kan zelfs een onervaren systeembeheerder gemakkelijk
de beveiliging bij de tijd houden.  Deze dienst is voor iedereen gratis
beschikbaar.}

\def\QualityCaption{Kwaliteit}

\def\Quality{Dankzij de persoonlijke inzet van de Debian ontwikkelaars
is de kwaliteit van de Debian-pakketten onge"evenaard.
Belangrijke hulp hierbij is de Debian Policy Manual,
waarin precies is beschreven hoe pakketten zich dienen te gedragen en op welke
manier die met het systeem samenwerken.  Een pakket dat zich
niet houdt aan deze policy zal niet geleverd worden in de offici"ele stabiele
Debian-versie. Verder wordt Debians publieke bugtrackingsysteem gebruikt
bij het bewaken van de kwaliteit.  Debian verbergt problemen niet.  Debian kent
geen commerci"ele druk en zal geen nieuwe, mogelijk onstabiele, versie
vrijgeven alleen omdat de markt dat vraagt.}

\def\IncludedCaption{Beschikbaar voor Debian GNU/Linux}

\def\Included{De complete Debian GNU/Linux-distributie is zo groot dat ze
minstens 12 DVD's, 74 CD's of 3 BD's nodig heeft. Deze disks zijn slechts de architectuurafhankelijke
programma's; de broncode komt daar nog bij. U vindt hierin ondermeer:}

\def\Utilities{de volledige set GNU hulpprogramma's, editors (o.a.~Emacs, vi),
netwerkprogrammatuur (o.a.~chat, filesharing),
webbrowsers (waaronder Chromium en ook Firefox en andere Mozilla producten),
privacysoftware (o.a.~gpg, ssh), emailprogramma's en vele
utilities;}

\def\Networking{een volledige set netwerkprotocollen (IPv4, IPv6, PPP,
SMB-netwerkomgeving,~\ldots);}

\def\Programming{ontwikkelingssoftware voor alle veelgebruikte programmeertalen
zoals C, C++, C\#, Objective-C, Java, Python, Perl, PHP, Ruby, en ook voor vele van de minder
gebruikte talen;}

\def\Windowsystem{het grafische X Window Systeem aangevuld met vele
windowmanagers en desktops zoals GNOME, MATE, KDE, Cinnamon, XFCE en LXDE;}

\def\Documents{het \TeX/\LaTeX\ documentsysteem, PostScript- en
Type1-lettertypen en hulpprogramma's, de GhostScript PostScript-interpreter en
een XML/SGML/HTML-ontwikkelomgeving;}

\def\Graphics{GIMP (een vrij alternatief voor Photoshop\tm);}

\def\Office{een complete set bureau-applicaties, waaronder de LibreOffice
productivitysuite, Gnumeric en andere spreadsheets, WYSIWYG-editors,
kalenders;}

\def\Databases{relationele databases, zoals PostgreSQL, MySQL, MariaDB, en ontwikkeltools
(applicatieservers, serverside scriptingtalen).}

\def\KnowMoreCaption{Meer weten?}

\def\KnowMore{Kijk op \boxurl{http://www.debian.org/}. Voor informatie of hulp
kunt u terecht op het IRC-kanaal \#debian op \boxurl{irc.debian.org}, of op
de mailinglijsten van Debian. Zie
\boxurl{http://www.debian.org/MailingLists/subscribe} voor meer informatie.
Verder zijn er talloze andere gebruikersforums.}

\def\Install{Voor het installeren van Debian GNU/Linux kunt u een enkele
CD downloaden van \boxurl{http://www.debian.org/distrib/} en
daarmee een netwerkinstallatie starten; of een aantal Debian DVD's/CD's bestellen of
branden. Debian verkoopt zelf geen disks maar stelt offici"ele DVD/CD-images ter
beschikking die door verschillende verkopers geproduceerd en verkocht worden. Voor
details over de offici"ele DVD/CD-images kunt u terecht op
\boxurl{http://www.debian.org/CD/}.}

\def\MadeWith{Deze folder is op \today\ gemaakt\\
met \LaTeX\ op een Debian systeem.}

%\def\SponsoredBy{Het drukken van deze folder is gesponsord door}
\def\SponsoredBy{Het drukken van deze folder is mogelijk gemaakt door uw donaties.}
% for simple text, use \SponsorDonations in layout.tex

% URL of the sponsor
% \def\SponsorURL{http://murphy.nl/}
\def\SponsorURL{ad 1810}

% EPS file in sponsors/ subdirectory
% \def\SponsorLogo{murphy}
\def\SponsorLogo{ad1810}
