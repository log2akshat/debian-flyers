\selectlanguage{dutch}

\def\Universal{Het Universele Besturingssysteem}

\def\WhatIsDebianCaption{Wat is Debian GNU/Linux?}

\def\WhatIsDebian{
%
Debian is een vrij besturingssysteem voor uw computer. Een
besturingssysteem is de set van basisprogramma's en hulpprogramma's
die uw computer in staat stellen om nuttige dingen te doen. Het hart van
elk besturingssysteem is de kernel. Die vormt het meest fundamentele
programma op uw computer: het doet de basishuishouding en stelt je in
staat om andere programma's te starten. Op dit moment is Debian
gebaseerd op de Linux kernel, en bevat het meer dan 9.500 pakketten met
allerlei programma's. Bijna 1.000 ontwikkelaars werken hard
om de hoge kwaliteit van Debian te bewaren.
%
}

\def\FreedomCaption{Vrijheid}

\def\Freedom{
%
Debian bestaat volledig uit Vrije Software. Da's meer dan
alleen gratis software; ook de vrijheid om de software voor elk doel
te gebruiken, te delen met je vrienden, de broncode
te lezen en aan te passen, en die wijzigingen aan andere mensen door te
geven. Daarom kan Debian gebruikt worden zonder beperkingen -
zelfs in commerci�le omgevingen! Debian is de grootste verzameling
Vrije Software, klaar voor gebruik, op het Internet.
%
}

\def\CommunityCaption{Community}

\def\Community{
%
Het Debian project wordt voor 100\% gedragen door
vrijwilligers, die zich inzetten om een wereldklasse Open
Source-besturingssysteem te produceren. Er zijn op dit moment zo'n
1.000 mensen overal ter wereld bezig met het ontwikkelen van het Debian
systeem, elk met taken vari�rend van het onderhouden van
pakketten tot quality assurance, security, policy, en
strategie. Het Debian project gelooft in de principes van
software-vrijheid en openheid, wat duidelijk wordt weergegeven in z'n
Social Contract, gepubliceerd op
\boxurl{http://www.debian.org/social_contract}. De Debian Free Software
Guidelines beschrijven de criteria waar licenties voor software in
het Debian besturingssysteem aan moeten voldoen. De Open
Source-definitie is een afgeleide van de Debian Free Software
Guidelines.
%
}

\def\ContinuityCaption{Continu�teit}

\def\Continuity{
%
Het Debian packaging-systeem laat een probleemloze overgang naar nieuwe
versies van software toe, zonder de vereiste om een nieuwe installatie
van nul af aan te beginnen, en zal uw oude configuratie niet
overschrijven. Afhankelijkheden worden automatisch
opgelost en U kan upgraden van alle mogelijke media: diskettes, harde
schijven, CD-ROMs, of direct vanaf het Internet.
%
}

\def\StabilityCaption{Stabiliteit}

\def\Stability{
%
Debian kent geen commerci�le druk en zal geen nieuwe, mogelijk
onstabiele, versie vrijgeven alleen omdat de markt dat
vereist. De Debian ontwikkelaars testen hun systeem altijd zeer uitgebreid
en proberen alle gekende bugs te verwijderen vooraleer een nieuwe versie
vrijgegeven wordt.
%
}

\def\PortabilityCaption{Overdraagbaarheid}

\def\Portability{
%
Debian is beschikbaar voor de volgende
architecturen: Alpha, ARM, HP PA-RISC, IBM S/390, Intel x86, Intel
IA-64, Motorola 68k, MIPS/MIPSel, PowerPC, SPARC.
%
}

\def\IncludedCaption{Beschikbaar voor Debian GNU/Linux}

\def\Included{
%
Eigenlijk is de complete Debian GNU/Linux-distributie zo groot dat ze
amper op zes CD's past (architectuur-afhankelijke voorgecompileerde
programma's, zelfs meer CD's met broncode). Daaronder zal U onder meer
vinden:
%
}

\def\Utilities{
%
de volledige set GNU hulpprogramma's, editors (emacs, vi, ~\ldots),
netwerkprogrammatuur (telnet, ftp, finger,~\ldots), webbrowsers, privacy
software (gpg, ssh,~\ldots), email clients, en zowat elk klein hulpprogramma
waar U aan kan denken.
%
}

\def\Networking{
%
een volledige set netwerkprotocols (PPP, TCP/IP, Apple\tm\ Ethertalk,
Windows\tm\ SMB, Novell\tm,~\ldots)
%
}

\def\Programming{
%
ontwikkelingssoftware voor de meestgebruikte programmeertalen (en zelfs
voor een aantal van de minder gebruikte talen) zoals: C, C++,
ObjectiveC, Java, Python, Perl, Smalltalk, LISP, Scheme, Haskell, ADA,
en meer.
%
}

\def\Windowsystem{
%
het X11 systeem aangevuld met tientallen window managers en de
twee leidende desktops: Gnome en KDE
%
}

\def\Documents{
%
het \TeX/\LaTeX\ documentsysteem, PostScript en Type1 lettertypen en
hulpprogramma's, de Ghostscript PostScript\tm\ interpreter, en een
complete XML/SGML/HTML ontwikkelomgeving
%
}

\def\Graphics{
%
the GIMP (GNU Image Manipulation Program), een vrij alternatief voor
Photoshop\tm.
%
}

\def\Office{
%
een complete set bureau-applicaties: WYSIWYG-editors, kalenders,
spreadsheets, databases, enz.
%
}

\def\Databases{
%
relationele databases, zoals PostgreSQL, MySQL en ontwikkeltools
(applicatieservers, server side scriptingtalen)
%
}

\def\KnowMoreCaption{Meer weten?}

\def\KnowMore{
%
Kijk op \boxurl{http://www.debian.org/}. Als je
informatie of hulp nodig hebt kan je terecht op het IRC kanaal
\#debian op \boxurl{irc.debian.org}, of op ��n van de Debian
mailinglists. Zie \boxurl{http://www.debian.org/MailingLists/subscribe}
voor instructies.
%
}

\def\Install{
%
Als U Debian GNU/Linux wenst te installeren, dan kan U de
installatiediskettes downloaden op \boxurl{ftp://ftp.debian.org/} en
van daaruit een netwerkinstallatie doen, of een aantal Debian CDs
bestellen. Debian verkoopt zelf geen CDs maar stelt Offici�le CD
Images ter beschikking die door verschillende verkopers gedrukt en
verkocht worden. Voor details over de Offici�le CD Images kan U
terecht op \boxurl{http://www.debian.org/CD/}.
%
}
