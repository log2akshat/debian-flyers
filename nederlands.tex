%   Copyright (c) 2002 Software in the Public Interest, Inc.
%
%   This program is free software; you can redistribute it and/or modify
%   it under the terms of the GNU General Public License as published by
%   the Free Software Foundation; version 2 dated June, 1991.
%
%   This program is distributed in the hope that it will be useful,
%   but WITHOUT ANY WARRANTY; without even the implied warranty of
%   MERCHANTABILITY or FITNESS FOR A PARTICULAR PURPOSE.  See the
%   GNU General Public License for more details.
%
%   You should have received a copy of the GNU General Public License
%   along with this program;  if not, write to the Free Software
%   Foundation, Inc., 59 Temple Place - Suite 330, Boston, MA 02111, USA.

%   translation-check translation="1.9"

% Contributed by Wouter Verhelst <wouter@debian.org>
% Updated by Bas Zoetekouw <bas@debian.org>

\selectlanguage{dutch}

\def\Universal{Het Universele Besturingssysteem}

\def\WhatIsDebianCaption{Wat is Debian GNU/Linux?}

\def\WhatIsDebian{
%
Debian is een vrij besturingssysteem voor uw computer. Een
besturingssysteem is de set van programma's die
ervoor zorgen dat uw computer nuttige dingen kan doen.
De kernel is het hart van het OS en het meest fundamentele
programma op uw computer: het doet de basishuishouding en stelt de
gebruiker in staat om programma's te starten. Op dit moment is
Debian gebaseerd op de Linux kernel en bevat het meer dan 9500
pakketten met allerlei programma's. Bijna 1000 ontwikkelaars helpen
om de hoge kwaliteit van Debian te behouden.%
%
}

\def\FreedomCaption{Vrijheid}

\def\Freedom{
%
Debian bestaat volledig uit Vrije Software. Dat betekent meer dan alleen
dat de software gratis is: ook de vrijheid om de software voor elk doel
te gebruiken, te delen met uw vrienden en kennissen, de broncode te
lezen en aan te passen en die wijzigingen aan anderen door te geven,
valt eronder.  Debian kan dus gebruikt worden zonder beperkingen ---
zelfs in commerci\"ele omgevingen! Debian is de grootste verzameling
vrije software beschikbaar op het Internet en is direct klaar voor
gebruik.%
%
}

\def\CommunityCaption{Gemeenschap}

\def\Community{
%
Het Debian project wordt voor 100\% gedragen door
vrijwilligers, die zich inzetten om een wereldklasse
Open Source besturingssysteem te produceren. Er zijn op dit moment zo'n
duizend mensen overal ter wereld bezig met het ontwikkelen van Debian,
elk met taken vari\"erend van het onderhouden van pakketten tot quality
assurance, beveiliging, beleid en strategie. Het Debian project gelooft
in de principes van software-vrijheid en -openheid, hetgeen duidelijk
wordt weergegeven in het Sociale Contract, dat is gepubliceerd op
\boxurl{http://www.debian.org/social_contract}. De Debian Richtlijnen
voor Vrije Software beschrijven de criteria waaraan licenties voor
software in het Debian besturingssysteem moeten voldoen. De
algemene Open-Source--definitie is een afgeleide van de Debian
Richtlijnen voor Vrije Software.
%
}

\def\ContinuityCaption{Continu\"\i{}teit}

\def\Continuity{
%
Het packaging-systeem van Debian laat een probleemloze overgang naar
nieuwe versies van software toe, zonder dat u vanaf nul moet beginnen, 
en zonder dat uw oude configuratie
overschreven zal worden. Onderlinge afhankelijkheden tussen pakketten
worden automatisch opgelost en u kan upgraden van allerlei media:
diskettes, harde schijven, CDROM's en direct vanaf het Internet.
%
}

\def\StabilityCaption{Stabiliteit}

\def\Stability{
%
Debian kent geen commerci\"ele druk en zal geen nieuwe, mogelijk
onstabiele, versie vrijgeven, alleen omdat de markt dat vereist. De
ontwikkelaars van Debian  testen hun systeem altijd zeer uitgebreid en
proberen alle bekende bugs te verwijderen voordat een nieuwe versie
vrijgegeven wordt.
%
}

\def\PortabilityCaption{Compatibiliteit}

\def\Portability{
%
Debian is beschikbaar voor de volgende computerarchitecturen: Alpha,
ARM, HP PA-RISC, IBM S/390, Intel x86, Intel IA-64, Motorola 68k,
MIPS/MIPSel, PowerPC, SPARC.
%
}

\def\IncludedCaption{Beschikbaar voor Debian GNU/Linux}

\def\Included{
%
De complete Debian GNU/Linux-distributie zo groot dat ze nauwelijks op
zes CD's past (dit zijn slechts de architectuur-afhankelijke
programma's;  de CD's met broncode komen daar nog bij).
Op deze CD's vindt u onder meer:
%
}

\def\Utilities{
%
de volledige set GNU hulpprogramma's, editors (o.a.~emacs, vi),
netwerkprogrammatuur (o.a.~telnet, ftp), webbrowsers, privacy
software (o.a.~gpg, ssh), emailprogramma's en vele utilities
%
}

\def\Networking{
%
een volledige set netwerkprotocollen (PPP, TCP/IP, Apple\tm~Ethertalk,
Windows\tm~SMB, Novell\tm,~\ldots)
%
}

\def\Programming{
%
ontwikkelingssoftware voor de meest gebruikte programmeertalen (en
voor een aantal van de minder gebruikte talen) zoals: C, C++,
Objective-C, Java, Python, Perl, Smalltalk, LISP, Scheme, Haskell, ADA,
enzovoort.
%
}

\def\Windowsystem{
%
het X11 systeem aangevuld met tientallen windowmanagers en de
twee leidende desktops: Gnome en KDE
%
}

\def\Documents{
%
het \TeX/\LaTeX\ documentsysteem, PostScript- en Type1-lettertypen en
hulpprogramma's, de GhostScript PostScript\tm-interpreter en een
XML/SGML/HTML-ontwikkelomgeving
%
}

\def\Graphics{
%
the GIMP (GNU Image Manipulation Program), een (vrij) alternatief voor
Photoshop\tm.
%
}

\def\Office{
%
een complete set bureau-applicaties: WYSIWYG-editors, kalenders,
spreadsheets, databases, enzovoort
%
}

\def\Databases{
%
relationele databases, zoals PostgreSQL, MySQL en ontwikkeltools
(applicatieservers, server-side scriptingtalen)
%
}

\def\KnowMoreCaption{Meer weten?}

\def\KnowMore{
%
Kijk op \boxurl{http://www.debian.org/}. Als u informatie of hulp nodig
hebt, kunt u terecht op het IRC-kanaal \#debian op
\boxurl{irc.debian.org}, of op een van de mailinglijsten van Debian.
Zie \boxurl{http://www.debian.org/MailingLists/subscribe} voor meer
informatie.
%
}

\def\Install{
%
Als u Debian GNU/Linux wenst te installeren, dan kunt u de
installatiediskettes downloaden van \boxurl{ftp://ftp.debian.org/} en
van daaruit een netwerkinstallatie doen, of een aantal Debian CD's
bestellen. Debian verkoopt zelf geen CD's maar stelt offici\"ele
CD-images ter beschikking die door verschillende verkopers geperst en
verkocht worden. Voor details over de offici\"ele CD-images kunt u terecht
op \boxurl{http://www.debian.org/CD/}.%
}

\def\MadeWith{
%
Deze folder is gemaakt met \LaTeX\ op een Debian systeem.
%
}

\def\SponsoredBy{
%
Het drukken van deze folder is gesponsord door
%
}

% URL of the sponsor
\def\SponsorURL{http://www.credativ.de/}

% EPS file in sponsors/ subdirectory
\def\SponsorLogo{credativ}
