%   Copyright (c) 2002 Software in the Public Interest, Inc.
%
%   This program is free software; you can redistribute it and/or modify
%   it under the terms of the GNU General Public License as published by
%   the Free Software Foundation; version 2 dated June, 1991.
%
%   This program is distributed in the hope that it will be useful,
%   but WITHOUT ANY WARRANTY; without even the implied warranty of
%   MERCHANTABILITY or FITNESS FOR A PARTICULAR PURPOSE.  See the
%   GNU General Public License for more details.
%
%   You should have received a copy of the GNU General Public License
%   along with this program;  if not, write to the Free Software
%   Foundation, Inc., 59 Temple Place - Suite 330, Boston, MA 02111, USA.

%   translation-check translation="1.16"

% Contributed by Daniele Medri <madrid@linux.it>
% and Luca Monducci <luca.mo@tiscali.it>

\selectlanguage{italian}

\def\MyRCS{\rcsInfo $Id: italiano.tex,v 1.12 2008/05/24 09:47:30 joostvb Exp $}

\def\Universal{Il sistema operativo universale}

\def\WhatIsDebianCaption{Cos'� Debian GNU/Linux?}

\def\WhatIsDebian{
%
Debian � un sistema operativo Libero per il tuo computer. Debian �
attualmente basato sul kernel Linux e include pi� di 10.000 pacchetti
tra utilit� e applicazioni, molto pi� software di qualsiasi altro sistema
operativo Libero. Sono supportate parecchie architetture, comprese Intel
x86, Intel IA-64, AMD64, Alpha, Motorola 68k, PowerPC e Sun SPARC. Pi� di
1.000 volontari lavorano duro per mantenere l'alta qualit� di Debian.
%
}

\def\FreedomCaption{Libert�}

\def\Freedom{
%
Come sancito dal Contratto Sociale, Debian � composta esclusivamente da
Software Libero. Con Software Libero, non intendiamo solo l'assenza di
costi, ma la Libert� di utilizzarlo, condividerlo con gli amici, leggere
e modificare il codice sorgente e distribuire questi cambiamenti ad
altre persone.
%
}

\def\ContinuityCaption{Continuit�}

\def\Continuity{
%
Il sistema di gestione dei pacchetti Debian permette la transizione a
nuove versioni dei programmi senza richiedere una nuova installazione
e non elimina i vecchi file di configurazione. Le dipendenze tra i
programmi sono gestite automaticamente: se un pacchetto richiede altri
pacchetti, il programma di installazione se ne fa carico e puoi installare
e aggiornare utilizzando dischi, CD-ROM o tramite una connessione di rete.
%
}

\def\SecurityCaption{Sicurezza}

\def\Security{
%
Debian � intrinsecamente sicura e facilmente adattabile nel caso si abbia
bisogno di un livello di sicurezza particolarmente elevato, � facile
installare un sistema Debian con con pochissimo software. Quando viene
scoperto un problema di sicurezza il Debian Security Team rilascia,
generalmente entro 48 ore, i pacchetti con le correzioni. Con l'aiuto del
sistema di gestione dei pacchetti anche un amministratore inesperto pu� tener
aggiornato il sistema dal punto di vista della sicurezza. Questo servizio
� disponibile gratuitamente per tutti.
%
}

\def\QualityCaption{Qualit�}

\def\Quality{
%
Poich� sono i singoli sviluppatori a inviare i pacchetti la loro qualit�
� ineguagliabile. Per tenere alto questo standard ci si avvale del Policy
Manual, che descrive esattamente come i pacchetti devono comportarsi e
interagire con il sistema. Un pacchetto che viola la Policy non pu� essere
incluso nella release stabile ufficiale. Un altro aiuto per il controllo
di qualit� � il sistema di gestione dei bug aperti, Debian non nasconde i
problemi. Debian non ha pressioni commerciali e non rilascer� una nuova
versione potenzialmente instabile solo perch� il mercato la richiede.
%
}

\def\IncludedCaption{Incluso in Debian GNU/Linux}

\def\Included{
%
Attualmente, la distribuzione completa di Debian GNU/Linux � composta da
almeno 2 DVD o da una dozzina di CD. Questi dischi contengono solo i
binari precompilati e vincolati all'architettura, servono ulteriori CD
per i sorgenti. All'interno si trovano:
%
}

\def\Utilities{
%
L'insieme completo delle utilit� GNU, editor (Emacs, vi,~\ldots), client
di rete (chat, filesharing,~\ldots), browser web (inclusi i browser
Mozilla), strumenti per la riservatezza (gpg, ssh,~\ldots), client di
posta elettronica e qualsiasi strumento di cui si abbia bisogno
%
}

\def\Networking{
%
Un insieme completo di protocolli di rete (PPP, IPv4, IPv6, SMB,~\ldots)
%
}

\def\Programming{
%
Strumenti di sviluppo per i principali linguaggi di programmazione come:
C, C++, Objective-C, Java, Python, Perl e altri linguaggi meno conosciuti
%
}

\def\Windowsystem{
%
Il sistema grafico X11 completo con dozzine di window manager e due
importanti ambienti desktop: GNOME e KDE
%
}

\def\Documents{
%
Il sistema di preparazione documenti \TeX/\LaTeX\, caratteri e strumenti
PostScript\tm\ e Type1, l'interprete PostScript\tm\ Ghostscript e un
ambiente di sviluppo completo XML/SGML/HTML
%
}

\def\Graphics{
%
GIMP, GNU Image Manipulation Program (una alternativa libera a Photoshop\tm)
%
}

\def\Office{
%
Un insieme completo di applicazioni per l'ufficio: OpenOffice.org,
Gnumeric e altri fogli di calcolo, editor WYSIWYG, calendari
%
}

\def\Databases{
%
Database relazionali, come PostgreSQL, MySQL e strumenti di sviluppo
(application server, linguaggi di scripting server side)
%
}

\def\KnowMoreCaption{Vuoi saperne di pi�?}

\def\KnowMore{
%
Visita il sito \boxurl{http://www.debian.org/}. Per maggiori informazioni
o aiuto, puoi entrare nel canale IRC \boxurl{\#debian} su
\boxurl{irc.debian.org}, oppure chiedere in una delle mailing list Debian.
Consulta le istruzioni su \boxurl{http://www.debian.org/MailingLists/subscribe}.
In pi� ci sono molti forum non ufficiali frequentati da altri utenti.
%
}

\def\Install{
%
Per installare Debian GNU/Linux si possono scaricare i CD da
\boxurl{ftp://ftp.debian.org/} e poi procedere all'installazione via
rete oppure ordinare (o masterizzare) i DVD/CD Debian. Debian non vende
i supporti per l'installazione, ma provvede alle immagini dei DVD/CD
ufficiali che molti distributori masterizzano e vendono. Per i dettagli
circa le immagini dei DVD/CD ufficiali visita
\boxurl{http://www.debian.org/CD/}.
%
}

\def\MadeWith{
%
Volantino creato il \today con \LaTeX\ e un sistema Debian.
%
}

\def\SponsoredBy{
%
Per la stampa del volantino si ringrazia
%
}

% URL of the sponsor
\def\SponsorURL{http://www.credativ.de/}

% EPS file in sponsors/ subdirectory
\def\SponsorLogo{credativ}

