%   Copyright (c) 2002 Software in the Public Interest, Inc.
%
%   This program is free software; you can redistribute it and/or modify
%   it under the terms of the GNU General Public License as published by
%   the Free Software Foundation; version 2 dated June, 1991.
%
%   This program is distributed in the hope that it will be useful,
%   but WITHOUT ANY WARRANTY; without even the implied warranty of
%   MERCHANTABILITY or FITNESS FOR A PARTICULAR PURPOSE.  See the
%   GNU General Public License for more details.
%
%   You should have received a copy of the GNU General Public License
%   along with this program;  if not, write to the Free Software
%   Foundation, Inc., 59 Temple Place - Suite 330, Boston, MA 02111, USA.

%   translation-check translation="1.9"

% Contributed by Daniele Medri <madrid@linux.it>

\selectlanguage{italian}

\def\Universal{Il sistema operativo universale}

\def\WhatIsDebianCaption{Che cos'� Debian GNU/Linux?}

\def\WhatIsDebian{
%
Debian � un sistema operativo (OS) free per il tuo computer. Un sistema
operativo � un insieme di programmi basilari e utilit� che gestiscono
un computer. Al centro di ogni sistema operativo vi � il kernel. E'
il programma fondamentale per un computer: si occupa di gestire
le basilari operazioni e permette di eseguire altri programmi. Debian �
attualmente basata sul kernel di Linux e include pi� di 9.500 pacchetti
tra utilit� e applicazioni. Pi� di 1.000 sviluppatori lavorano duro
per mantenere l'alta qualit� di Debian.
%
}

\def\FreedomCaption{Libert�}

\def\Freedom{
%
Debian � composta esclusivamente da Free Software. Per Free Software,
non intendiamo solo l'assenza di costi, ma la Libert� di utilizzarlo, condividerlo
con gli amici, leggere e modificare il codice sorgente e distribuire questi cambiamenti
ad altre persone. Questo significa che Debian pu� essere utilizzata senza limitazioni -
anche in ambienti commerciali! Debian � la pi� grande collezione di Free Software
pronto per essere installato su Internet.
%
}

\def\CommunityCaption{Comunit�}

\def\Community{
%
Il progetto Debian � rappresentato al 100\% da contributi volontari
atti a produrre un sistema operativo Open Source di prima classe. Ci
sono attualmente circa 1.000 persone da tutte le parti del mondo che
sviluppano il sistema Debian, ognuno con un ruolo che spazia dalla
realizzazione dei pacchetti al controllo di qualit�, sicurezza, policy
e strategia. Il progetto Debian � conforme ai principi della libert�
e apertura del software. E' chiaramente esposto nel contratto sociale
di Debian disponibile all'indirizzo \boxurl{http://www.debian.org/social_contract}.
Le linee guida Debian relative al Free Software descrivono i criteri che le
licenze dei software inclusi devono rispettare. La definizione di Open Source
� un lavoro derivato dalle linee guida Debian sul Free Software.
%
}

\def\ContinuityCaption{Continuit�}

\def\Continuity{
%
Il sistema di gestione dei pacchetti Debian permette la transizione
a nuove versioni dei programmi senza richiedere una nuova installazione e non
elimina le vecchie configurazioni. Le dipendenze tra i programmi sono gestite
automaticamente: se un pacchetto richiede altri pacchetti, il programma di
installazione se ne fa carico e puoi installare e aggiornare utilizzando dischi,
CD-rom o tramite la connessione di rete.
%
}

\def\StabilityCaption{Stabilit�}

\def\Stability{
%
Debian non ha pressioni commerciali e non rilascer� una nuova versione instabile
solo perch� il mercato la richiede. I maintainer di Debian controllano sempre il
sistema attentamente e provvedono a correggere tutti i bug conosciuti.
%
}

\def\PortabilityCaption{Portabilit�}

\def\Portability{
%
Debian � disponibile e funziona equamente bene sulle seguenti architetture:
Alpha, ARM, HP PA-RISC, IBM S/390, Intel x86,
Intel IA-64, Motorola 68k, MIPS/MIPSel, PowerPC, SPARC.
%
}

\def\IncludedCaption{Inclusi con Debian GNU/Linux}

\def\Included{
%
Attualmente, la distribuzione completa di Debian GNU/Linux occupa 6 CD
(binari precompilati e vincolati all'architettura, e pi� CD con il codice sorgente.).
All'interno puoi trovare:
%
}

\def\Utilities{
%
un insieme completo di utilit� GNU, editor (emacs, vi,~\ldots), client di rete
 (telnet, ftp, finger,~\ldots), navigatori web, strumenti per la riservatezza
(gpg, ssh,~\ldots), client di posta elettronica
%
}

\def\Networking{
%
un insieme completo di protocolli di rete (PPP, TCP/IP, Apple\tm\ EtherTalk,
Windows\tm\ SMB, Novell\tm,~\ldots)
%
}

\def\Programming{
%
strumenti di sviluppo per i principali linguaggi di programmazione (e anche
per quelli meno conosciuti) come: C, C++, Objective-C, Java, Python,
Perl, Smalltalk, Lisp, Scheme, Haskell, ADA, e altro ancora
%
}

\def\Windowsystem{
%
il sistema X11 completo con dozzine di window manager e due importanti
ambienti desktop: GNOME e KDE
%
}

\def\Documents{
%
il sistema di preparazione documenti \TeX/\LaTeX\, caratteri e strumenti PostScript
e Type1, l'interprete Ghostscript Postscript e un ambiente di sviluppo completo
XML/SGML/HTML.
%
}

\def\Graphics{
%
GIMP, GNU Image Manipulation Program (una libera alternativa a Photoshop\tm)
%
}

\def\Office{
%
un insieme completo di applicazioni per l'ufficio: editor WYSIWYG,
calendari, fogli di calcolo, database, ecc.
%
}

\def\Databases{
%
database relazionali, come PostgreSQL, MySQL e strumenti di sviluppo
(application server, linguaggi server side scripting)
%
}

\def\KnowMoreCaption{Vuoi saperne di pi�?}

\def\KnowMore{
%
Punta semplicemente il tuo navigatore su
\boxurl{http://www.debian.org/}.
Se hai bisogno di informazioni o aiuto, puoi andare su IRC ed entrare nei
canali \#debian o \boxurl{irc.debian.org}, o una delle mailing list Debian.
Visita \boxurl{http://www.debian.org/MailingLists/subscribe} per le istruzioni.
%
}

\def\Install{
%
Se desideri installare Debian GNU/Linux, puoi scaricare i floppy da
\boxurl{ftp://ftp.debian.org/} e poi procedere all'installazione via rete,
o ordinare alcuni CD Debian. Debian non vende CD, ma provvede alle immagini
ufficiali che molti venditori masterizzano e vendono. Per dettagli circa le
immagini dei CD ufficiali visita \boxurl{http://www.debian.org/CD/}.
%
}

\def\MadeWith{
%
This flyer was made using \LaTeX\ and a Debian system.
%
}

\def\SponsoredBy{
%
Printing of this flyer was sponsored by
%
}

% URL of the sponsor
\def\SponsorURL{http://www.credativ.de/}

% EPS file in sponsors/ subdirectory
\def\SponsorLogo{credativ}
