%   Copyright (c) 2000,2002 Software in the Public Interest, Inc.
%
%   This program is free software; you can redistribute it and/or modify
%   it under the terms of the GNU General Public License as published by
%   the Free Software Foundation; version 2 dated June, 1991.
%
%   This program is distributed in the hope that it will be useful,
%   but WITHOUT ANY WARRANTY; without even the implied warranty of
%   MERCHANTABILITY or FITNESS FOR A PARTICULAR PURPOSE.  See the
%   GNU General Public License for more details.
%
%   You should have received a copy of the GNU General Public License
%   along with this program;  if not, write to the Free Software
%   Foundation, Inc., 59 Temple Place - Suite 330, Boston, MA 02111, USA.

%   translation-check translation="1.13"

\selectlanguage{french}

\def\MyRCS{\rcsInfo $Id: francaise.tex,v 1.14 2011/02/04 10:58:51 gismo Exp $}

\def\Universal{Le système d'exploitation universel}

\def\WhatIsDebianCaption{Qu'est-ce que Debian GNU/Linux?}

\def\WhatIsDebian{
%
Debian est un système d'exploitation libre pour votre ordinateur. Debian est
basée sur le noyau Linux et inclut plus de 42~000~paquets de
logiciels et utilitaires, soit bien plus que n'importe quel autre système
d'exploitation libre. Dix architectures et variantes sont gérées, dont les
PC 32 et 64~bits. Depuis 1993, des développeurs volontaires (actuellement
plus de 10~000) travaillent pour maintenir Debian à un haut niveau de
qualité.
%
}

\def\FreedomCaption{Liberté}

\def\Freedom{
%
Comme indiqué dans le \og Contrat Social Debian \fg, Debian est entièrement
composée de logiciels libres. Par logiciel libre, nous ne voulons pas dire
uniquement gratuit, mais dire aussi la liberté de l'utiliser comme bon vous
semble, de le partager avec vos amis, d'en lire et modifier le code source
et de redistribuer ces changements.
%
}

\def\ContinuityCaption{Continuité}

\def\Continuity{
%
Le système de paquetage de Debian permet une migration facile vers de
nouvelles versions de programmes sans exiger de réinstallation depuis zéro,
et sans écraser votre ancienne configuration. Les dépendances entre
programmes sont gérées automatiquement: si un logiciel que vous voulez
installer requiert un autre logiciel, le programme d'installation s'en
occupera. Vous pouvez installer et mettre à jour Debian en utilisant des disques
Blu-ray, DVD, CD, une clef USB, ou une connexion réseau selon votre
convenance.
%
}

\def\SecurityCaption{Sécurité}

\def\Security{
%
Debian est sécurisée par défaut, et facilement adaptable au cas où un niveau
de sécurité très élevé serait nécessaire. Il est facile d'installer un
système Debian avec très peu de logiciels. Quand un problème de sécurité est
trouvé, l'équipe de sécurité de Debian fournit des logiciels corrigés,
généralement dans les 48~h. En matière de sécurité, il est donc très facile
pour un administrateur inexpérimenté de se tenir à jour grâce au système de
gestion de paquets de Debian. Ce service est disponible gratuitement pour tous.
%
}

\def\QualityCaption{Qualité}

\def\Quality{
%
Grâce à l'engagement personnel des développeurs Debian, la qualité des
paquetages est unique. Maintenir ce haut niveau de qualité est facilité par
la Charte Debian, qui décrit exactement comment les logiciels empaquetés
doivent se comporter et interagir avec le système. Un paquetage qui viole la
Charte Debian ne sera pas inclus dans la version stable officielle de
Debian. Une autre atout pour maintenir le niveau de qualité est le système
ouvert de gestion de bogues de Debian. Debian ne cache pas les problèmes.
Debian ne subit aucune pression commerciale et ne sort pas une nouvelle
version potentiellement instable juste parce que le marché l'exige.
%
}




\def\IncludedCaption{Inclus dans Debian GNU/Linux}

\def\Included{
%
La distribution Debian GNU/Linux complète consiste en plus de 12~DVD, 74~CD
ou 3~disques Blu-ray. Ces supports contiennent uniquement les logiciels
précompilés pour les architectures, les sources étant disponibles sur
d'autres médias. Vous y trouverez:
%
}

\def\Utilities{
%
l'ensemble complet des utilitaires GNU, des éditeurs (Emacs, vi\ldots), des
clients réseaux (chat, partage de fichiers\ldots), des navigateurs web
(dont Chromium, Firefox et autres produits de Mozilla), des outils de vie
privée (gpg, ssh\ldots), des clients de courriel, et tous les petits outils
utiles imaginables;
%
}

\def\Networking{
%
l'ensemble des protocoles réseau (PPP, IPv4, IPv6, voisinage réseau
SMB\ldots);
%
}

\def\Programming{
%
des outils de développement pour les principaux langages de programmation,
tels que: C, C++, Objective-C, Java, Python, Perl, PHP, Ruby et une grande
partie des plus méconnus;
%
}

\def\Windowsystem{
%
le système de fenêtrage X11 au complet, accompagné de dizaines de
gestionnaires de fenêtres et des principaux bureaux: GNOME,
MATE, KDE, Cinnamon, XFCE et LXDE;
%
}

\def\Documents{
%
le système de préparation de document \TeX/\LaTeX\thinspace, des fontes et
outils PostScript\tm\ et Type1, l'interpréteur PostScript\tm\
Ghostscript\tm\ et un environnement de développement complet XML/SGML/HTML;
%
}

\def\Graphics{
%
GIMP, le Programme de Manipulation d'Images de GNU (une alternative libre à
Photoshop\tm);
%
}

\def\Office{
%
un ensemble complet d'applications bureautiques, y compris la suite
LibreOffice, Gnumeric et d'autres tableurs, des éditeurs
WYSIWYG, des agendas;
%
}

\def\Databases{
%
des bases de données relationnelles comme PostgreSQL, MySQL et MariaDB, des
outils de développement (serveurs d'applications et langages de scripts
orientés serveur).
%
}

\def\KnowMoreCaption{Vous voulez en savoir plus?}

\def\KnowMore{
%
Pointez votre navigateur sur \boxurl{http://www.debian.org/}. Pour plus de
renseignements ou de l'aide, vous pouvez joindre le canal IRC \#debian sur
\boxurl{irc.debian.org} ou l'une des listes de diffusion de Debian
\boxurl{http://www.debian.org/MailingLists/subscribe}. De plus, il existe de
nombreux forums non officiels.
%
}

\def\Install{
%
Si vous désirez installer Debian GNU/Linux, vous pouvez depuis
\boxurl{http://www.debian.org/distrib/} télécharger un seul CD
d'installation et l'utiliser pour démarrer une installation par le réseau,
ou commander (ou graver) quelques CD ou DVD de Debian. Debian ne vend pas de
CD directement, mais fournit des images de CD et DVD officielles que de
nombreux vendeurs pressent et vendent. Pour les détails à propos des images
de CD et DVD officielles, allez simplement à
\boxurl{http://www.debian.org/CD/}.
%
}

\def\MadeWith{
%
Ce prospectus a été créé le \today\\
avec \LaTeX\ sur un système Debian.
%
}

%\def\SponsoredBy{Printing of this flyer was sponsored by}
\def\SponsoredBy{L'impression de ce prospectus a été rendue possible grâce à
vos dons.}
% for simple text, use \SponsorDonations in layout.tex

% URL of the sponsor
% \def\SponsorURL{http://www.credativ.de/}
\def\SponsorURL{http://www.powercraft.nl/}
% \def\SponsorURL{http://www.ffis.de/}

% EPS file in sponsors/ subdirectory
% \def\SponsorLogo{credativ}
\def\SponsorLogo{powercraft}
% \def\SponsorLogo{ffis-logo-color}
